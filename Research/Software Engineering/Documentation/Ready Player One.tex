%Custom Functions
\newcommand{\CompanyName}{TT} % update later

\documentclass[conference]{IEEEtran}
\IEEEoverridecommandlockouts
% The preceding line is only needed to identify funding in the first footnote. If that is unneeded, please comment it out.
%\usepackage{cite}
\usepackage{amsmath,amssymb,amsfonts}
\usepackage{algorithmic}
\usepackage{graphicx}
\usepackage{textcomp}
\usepackage{xcolor}

\usepackage{pdflscape}

\usepackage[utf8]{inputenc}
\usepackage{fancyhdr}
\usepackage{lastpage}

% Please add the following required packages to your document preamble:
\usepackage{multirow}
\usepackage[numbers]{natbib}

\usepackage{listings}
\usepackage{hyperref}
\usepackage{amsmath}

\hypersetup{
    colorlinks=true,
    linkcolor=black,
    filecolor=magenta,      
	urlcolor=cyan
}

\usepackage{listings}
\usepackage{color}

\definecolor{dkgreen}{rgb}{0,0.6,0}
\definecolor{gray}{rgb}{0.5,0.5,0.5}
\definecolor{mauve}{rgb}{0.58,0,0.82}

\lstset{frame=single,
  language=C++,
  showstringspaces=false,
  columns=flexible,
  basicstyle={\small\ttfamily},
  numbers = none,
  numberstyle=\tiny\color{gray},
  keywordstyle=\color{blue},
  commentstyle=\color{dkgreen},
  stringstyle=\color{mauve},
  breaklines=true,
  breakatwhitespace=true,
  tabsize=2
}

\def\BibTeX{{\rm B\kern-.05em{\sc i\kern-.025em b}\kern-.08em
T\kern-.1667em\lower.7ex\hbox{E}\kern-.125emX}}

\fancypagestyle{fancylandscape}{
\fancyhf{} %Clears the header/footer
\fancyfoot{% Footer
\makebox[\textwidth][r]{% Right
  \rlap{\hspace{.75cm}% Push out of margin by \footskip
    \smash{% Remove vertical height
      \raisebox{4.87in}{% Raise vertically
        \rotatebox{90}{Page \thepage\ of \pageref{LastPage}}}}}}}% Rotate counter-clockwise
\renewcommand{\headrulewidth}{0pt}% No header rule
\renewcommand{\footrulewidth}{0pt}% No footer rule
}

\pagestyle{fancyplain}
\fancyhf{}
\fancyfoot[c]{Page \thepage\ of \pageref{LastPage}}
\renewcommand{\headrulewidth}{0pt}

\begin{document}

	\title{Ready Player One - Research Document}

	\author{\IEEEauthorblockN{1\textsuperscript{st} Given Edward Patch}
	\IEEEauthorblockA{\textit{Software Engineer (of BSc Year 3)} \\
    \textit{Cloud Computing}\\
    \textit{University of Wales Trinity St. Davids (of Nabeel Masih)}\\
    Swansea, Wales \\
    Student ID: 1801492}}

     \maketitle
    
    \thispagestyle{plain}
    \pagestyle{plain}
    
    \tableofcontents
	  \vspace{.5cm}
    %\newpage
    \begin{abstract}
      The research document covers the ideology of the Metaverse. The research perspective is from a Software Engineer role to see how the technology and how Software Engineering may change in the future.
    \end{abstract}

    \begin{IEEEkeywords}
      Metaverse, Collective, Virtual Reality, VirtualSpace
    \end{IEEEkeywords}

    \section{Introduction}
      The Metaverse idea was introduced as early as 1992 by Neal Stephenson in the Science-Fiction Novel `Snow Crash'. %This 29-year-old book predicted the 'metaverse' - CNBChttps://www.cnbc.com › 2021/11/03 › how-the-1992-sci-f...
        
      To make us think that the Metaverse is applicable in 50 years of today in universities has Mark Zuckerberg has recently proposed to change the name of Facebook to Meta. By taking a look at %https://about.facebook.com/meta?utm_source=Google&utm_medium=paid-search&utm_campaign=metaverse&utm_content=post-launch 
      to understand Mark Zuckerberg's intention with Meta.

      Essentially, the Metaverse is a platform, which joins other platform's together, less realistically speaking, similar to the Matrix in a 3D environment. The way the Matrix works, in theory, is that the human body is in a container and lives in a dying world, whilst living in a different reality, a virtual environment. However, the proposed Metaverse we are looking at will not be unethical or as advanced as the Matrix movies.

      The research document will cover the ethical issues, obstacles and benefits that the Metaverse may encounter, covering how the implementation, user interaction (UI), and user experience (UX) are covered.
    \section{Literature Review}
      \subsection{Ethical Issues}

      \subsection{Advntages and Disadvantages}

    \section{Implmentation}
      \subsection{User Interaction}
      \subsection{User Experience}
      
    \section{Libraries and Resources}

    \section{Conclusion}

    \section{Terminology}
      List of terminologies used in this document:-
      \begin{itemize}
        \item UI - User Interaction.
        \item UX - User Experience.
      \end{itemize}

  \nocite{*}
	\renewcommand\refname{\section{Reference List}}
	\small{\bibliographystyle{IEEEtran}
    \bibliography{ref}}
\end{document}